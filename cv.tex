%-----------------------------------------------------------------------------------------------------------------------------------------------%
%	The MIT License (MIT)
%
%	Copyright (c) 2021 Jitin Nair
%
%	Permission is hereby granted, free of charge, to any person obtaining a copy
%	of this software and associated documentation files (the "Software"), to deal
%	in the Software without restriction, including without limitation the rights
%	to use, copy, modify, merge, publish, distribute, sublicense, and/or sell
%	copies of the Software, and to permit persons to whom the Software is
%	furnished to do so, subject to the following conditions:
%	
%	THE SOFTWARE IS PROVIDED "AS IS", WITHOUT WARRANTY OF ANY KIND, EXPRESS OR
%	IMPLIED, INCLUDING BUT NOT LIMITED TO THE WARRANTIES OF MERCHANTABILITY,
%	FITNESS FOR A PARTICULAR PURPOSE AND NONINFRINGEMENT. IN NO EVENT SHALL THE
%	AUTHORS OR COPYRIGHT HOLDERS BE LIABLE FOR ANY CLAIM, DAMAGES OR OTHER
%	LIABILITY, WHETHER IN AN ACTION OF CONTRACT, TORT OR OTHERWISE, ARISING FROM,
%	OUT OF OR IN CONNECTION WITH THE SOFTWARE OR THE USE OR OTHER DEALINGS IN
%	THE SOFTWARE.
%	
%
%-----------------------------------------------------------------------------------------------------------------------------------------------%

%----------------------------------------------------------------------------------------
%	DOCUMENT DEFINITION
%----------------------------------------------------------------------------------------

% article class because we want to fully customize the page and not use a cv template
\documentclass[a4paper,10pt]{article}

%----------------------------------------------------------------------------------------
%	FONT
%----------------------------------------------------------------------------------------

% % fontspec allows you to use TTF/OTF fonts directly
% \usepackage{fontspec}
% \defaultfontfeatures{Ligatures=TeX}

% % modified for ShareLaTeX use
% \setmainfont[
% SmallCapsFont = Fontin-SmallCaps.otf,
% BoldFont = Fontin-Bold.otf,
% ItalicFont = Fontin-Italic.otf
% ]
% {Fontin.otf}

%----------------------------------------------------------------------------------------
%	PACKAGES
%----------------------------------------------------------------------------------------
\usepackage{url}
\usepackage{parskip} 	

%other packages for formatting
\RequirePackage{color}
\RequirePackage{graphicx}
\usepackage[usenames,dvipsnames]{xcolor}
\usepackage[scale=0.9]{geometry}

%package for Russsian language
\usepackage[utf8]{inputenc}
\usepackage[russian]{babel}

%tabularx environment
\usepackage{tabularx}

%for lists within experience section
\usepackage{enumitem}

% centered version of 'X' col. type
\newcolumntype{C}{>{\centering\arraybackslash}X} 

%to prevent spillover of tabular into next pages
\usepackage{supertabular}
\usepackage{tabularx}
\newlength{\fullcollw}
\setlength{\fullcollw}{0.47\textwidth}

%custom \section
\usepackage{titlesec}				
\usepackage{multicol}
\usepackage{multirow}

%CV Sections inspired by: 
%http://stefano.italians.nl/archives/26
\titleformat{\section}{\Large\scshape\raggedright}{}{0em}{}[\titlerule]
\titlespacing{\section}{0pt}{10pt}{5pt}

%for publications
\usepackage[style=authoryear,sorting=ynt, maxbibnames=2]{biblatex}

%Setup hyperref package, and colours for links
\usepackage[unicode, draft=false]{hyperref}
\definecolor{linkcolour}{rgb}{0.7,0,0.25}
\hypersetup{colorlinks,breaklinks,urlcolor=linkcolour,linkcolor=linkcolour}
\addbibresource{citations.bib}
\setlength\bibitemsep{1em}

%for social icons
\usepackage{fontawesome5}

%debug page outer frames
%\usepackage{showframe}

% skill box
\usepackage{tcolorbox}
\newtcbox{\xmybox}[1][red]{on line,
arc=7pt,colback=#1!10!white,colframe=#1!50!black,
before upper={\rule[-3pt]{0pt}{10pt}},boxrule=1pt,
boxsep=0pt,left=6pt,right=6pt,top=2pt,bottom=1pt}

%----------------------------------------------------------

%----------------------------------------------------------------------------------------
%	BEGIN DOCUMENT
%----------------------------------------------------------------------------------------
\begin{document}

% non-numbered pages
\pagestyle{empty} 

%----------------------------------------------------------------------------------------
%	TITLE
%----------------------------------------------------------------------------------------


\begin{tabularx}{\linewidth}{@{} C @{}}
\Huge{Вероника Царева} \\[7.5pt]
\href{https://github.com/veronikatsareva}{\raisebox{-0.05\height}\faGithub\ veronikatsareva} \ $|$ \ 
\href{mailto:vveronikatsareva@yandex.ru}{\raisebox{-0.05\height}\faEnvelope \ vveronikatsareva@yandex.ru} \ $|$ \ 
\href{https://t.me/fromdeath2morning}{\raisebox{-0.05\height}\faTelegram \ @fromdeath2morning} \
\end{tabularx}

%----------------------------------------------------------------------------------------
%	EDUCATION
%----------------------------------------------------------------------------------------
\section{Образование}
\begin{tabularx}{\linewidth}{@{}l X@{}}	
\textbf{НИУ ВШЭ, Москва} & \hfill \normalsize Сентябрь 2022 –– н.в.\\
<<Фундаментальная и компьютерная лингвистика>>, бакалавриат. & \hfill \normalsize GPA: 8.53 / 10
\end{tabularx}

\underline{Релевантные курсы:} Математический анализ и линейная алгебра (9/10), Теория вероятностей и математическая статистика (8/10), Программирование: веб-сервисы и основы автоматической обработки языка (10/10), Автоматическая обработка естественного языка (10/10), Введение в анализ данных для гуманитарных и социальных наук на R (9/10), Анализ данных для лингвистов (10/10).

%----------------------------------------------------------------------------------------
% PROJECT SECTIONS
%----------------------------------------------------------------------------------------
\section{Проекты}

\begin{tabularx}{\linewidth}{ @{}l r@{} }

\textbf{Balanced Language Sampling for Multilingual Models | 2025 | Исследовательский проект} & \hfill \href{https://github.com/veronikatsareva/BalancedLanguageSampling}{Ссылка на GitHub} \\ [1.75pt]
\xmybox[purple]{Python} \xmybox[purple]{mBERT} \xmybox[purple]{NLP} \\ [3.75pt]
\multicolumn{2}{@{}X@{}}{A term paper that analyzes the results of the mBERT depending on the language sample (balanced/unbalanced) was used.} \\ [5.75pt]

\textbf{MedCorpora | 2024 | Групповой проект} & \hfill \href{https://github.com/veronikatsareva/MedCorpora}{Ссылка на GitHub}  \\ [1.75pt]
\xmybox[purple]{Python} \xmybox[purple]{Stanza} \xmybox[purple]{NLP} \\ [3.75pt]
\multicolumn{2}{@{}X@{}}{Веб-сайт, поддерживающий поиск по автоматически размеченному по частям речи корпусу медицинских новостей.} \\ [5.75pt]

\textbf{Organization's Network Graph | 2024 | Индивидуальный проект} & \hfill \href{https://github.com/veronikatsareva/math4ling-lab-1}{Ссылка на GitHub} \\[1.75pt]
\xmybox[purple]{Python} \xmybox[purple]{Networkx} \xmybox[purple]{NLP} \\ [3.75pt]
\multicolumn{2}{@{}X@{}}{Построение графа социальной сети по названиям организаций из текстов новостей и подсчет степени посредничества.} \\ [5.75pt]

\textbf{PROTOtypes | 2024 | Индивидуальный проект} & \hfill \href{https://github.com/veronikatsareva/FlaskPrototypesApp}{Ссылка на GitHub} \\[1.75pt]
\xmybox[purple]{Python} \xmybox[purple]{HTML} \xmybox[purple]{CSS} \xmybox[purple]{Graphs} \\ [3.75pt]
\multicolumn{2}{@{}X@{}}{Веб-сайт, собирающий прототипические сущности среди носителей русского языка с помощью анкеты.} \\ [5.75pt]

\textbf{CuteCodeBot | 2023 | Групповой проект} & \hfill \href{https://gitlab.com/veronikatsareva/cute-code-bot}{Ссылка на GitLab} \\[1.75pt]
\xmybox[purple]{Python} \xmybox[purple]{API}  \\ [3.75pt]
\multicolumn{2}{@{}X@{}}{Телеграм-бот, кодирующий и декодирующий сообщения с использованием четырех шифров.} \\  [5.75pt]

\end{tabularx}

%----------------------------------------------------------------------------------------
%	SKILLS
%----------------------------------------------------------------------------------------
\section{Навыки}
\begin{tabularx}{\linewidth}{@{}l X@{}}
Языки &  \normalsize{Python, C\texttt{++}, R}\\
Иностранные языки  &  \normalsize{Английский язык (С1), Немецкий язык (B1)}\\  
Библиотеки & \normalsize{NumPy, Pandas, Plotly, Sklearn, SciPy, NLTK}\\
Технологии & \normalsize{Git, \LaTeX, MS Excel, SQL}\\
\end{tabularx}

%----------------------------------------------------------------------------------------
% EXPERIENCE SECTIONS
%----------------------------------------------------------------------------------------

%Interests/ Keywords/ Summary
%\section{Summary}
%This CV can also be automatically complied and published using GitHub Actions. For details, \href{https://github.com/jitinnair1/autoCV}{click here}.

%Experience
\section{Опыт работы}

\begin{tabularx}{\linewidth}{ @{}l r@{} }

\textbf{НИУ ВШЭ | Учебный ассистент} & \hfill Сентябрь 2023 –– н.в. \\ [1.75pt]
\multicolumn{2}{@{}X@{}}{Курс: <<Программирование и лингвистические данные>> 

\underline{Обязанности:} проведение онлайн-консультаций по вопросам учебной программы, составление задач для контрольных работ, проверка домашних заданий на плагиат в Яндекс.Контесте, заполнение ведомостей.}  \\
\end{tabularx}

%----------------------------------------------------------------------------------------
% OTHER ACTIVITIES
%----------------------------------------------------------------------------------------

%Other Activities
\section{Внеучебная деятельность}

\begin{tabularx}{\linewidth}{ @{}l r@{} }
\textbf{Репетитор по информатике и программированию на Python} & \hfill 2024 –– н.в. \\[3.75pt]

\textbf{Участница <<Баттла вузов. Кубок Y\&\&Y>> (Турнир по программированию среди студентов)} & \hfill 2024 –– 2025 \\[3.75pt]

\textbf{Куратор НИУ ВШЭ | ОП <<Фундаментальная и компьютерная лингвистика>>} & \hfill 2023 –– 2024 \\ [3.75pt]

\end{tabularx}

%----------------------------------------------------------------------------------------
%	PUBLICATIONS
%----------------------------------------------------------------------------------------
% \section{Publications}
% \begin{refsection}[citations.bib]
% \nocite{*}
% \printbibliography[heading=none]
% \end{refsection}

% \vfill
% \center{\footnotesize Last updated: \today}

\end{document}
